\documentclass{article}
\usepackage[T1]{fontenc}
\usepackage{csquotes}
\usepackage[brazilian]{babel}
\usepackage[biblatex,rascunhos,anonimizar,assuntos,sugestoes,comentarios]{hacksxexeo} 
\usepackage[normalem]{ulem}
\usepackage{outlines}
\usepackage{booktabs}
\usepackage{datetime}

%\usepackage[multiple]{footmisc}
\usepackage{hyperref}
\renewcommand{\labelitemi}{$\bullet$}
\renewcommand{\labelitemii}{$\circ$}
\renewcommand{\labelitemiii}{$\diamond$}
\renewcommand{\labelitemiv}{$\circle}

\addbibresource{bib.bib}

\title{Como vamos acabar a dissertação? v2.0}
\author{Geraldo Xexéo}
\date{\today\ - \ \currenttime}


\addtolength{\parskip}{0.4em}

\begin{document}
\maketitle
\tableofcontents
\section{Introdução}

Esse documento apresenta as regras para os alunos de 2017/1 que vão tentar acabar sua dissertação de mestrado, aproveitando o perdão da pandemia.

A última versão desse documento pode ser encontrada em \href{https://github.com/xexeo/Como-vamos-acabar-a-disserta-o-}{\url{https://github.com/xexeo/Como-vamos-acabar-a-disserta-o-}}

Não esqueça de ler o \textbf{Guia dos Orientados}\footnote{\url{http://www.xexeo.net/ensino/guia-dos-orientados/}}. Esse documento é uma extensão para esse momento de finalizar a dissertação, onde vamos usar algumas regras mais fortes.

Também não deixe de ler as instruções do pacote \href{https://github.com/xexeo/hacksxexeo/blob/main/dist/hacksxexeo.pdf}{\textbf{hacksxexeo}}\footnote{\url{https://github.com/xexeo/hacksxexeo/blob/main/dist/hacksxexeo.pdf}}.


\section{Nosso trabalho conjunto}

\subsection{O que será avaliado}

O andamento será da seguinte forma:
\begin{itemize}
    \item toda semana vamos nos encontrar, a menos de problemas pessoais;
    \item o texto da dissertação tem que evoluir toda a semana, a avaliação básica será sobre essa evolução;
    \item toda semana terá um acordo básico do que deve ser feito, que, mesmo que não explicitado, inclui evoluir o texto da dissertação;
    \item espera-se que o acordo seja cumprido, porém entende-se que alguns podem ser mais longos que 1 semana, e deverá ser cumprido parcialmente.
\end{itemize}

A avaliação  será feita em função de vocês \textbf{cumprirem ou não o ritmo de trabalho}.

\subsection{Estrutura de troca de informações}

Para que tudo funcione, será estabelecida uma estrutura específica para nossa colaboração, e que você deverá usar para escrever sua dissertação.

\begin{itemize}
    \item Você precisa \textbf{obrigatoriamente} ter compartilhado comigo:
    \begin{itemize}
        \item um \textbf{projeto Overleaf} usando o formato \textbf{CoppeTex} \footnote{\url{https://github.com/COPPE-UFRJ/CoppeTeX}}. Esse arquivo pode ser criado por mim para facilitar o compartilhamento;
        \item um \textbf{repositório GitHub} sincronizado com esse projeto, criado a partir do \textbf{Overleaf}, novamente pode criado por mim;
        \item um \textbf{repositório de código}, criado por você, no GitHub (eu prefiro) ou no GitLab;
        \item um \textbf{diretório no Google Drive}, com o nome do aluno ou aluna, criado por mim, para troca de artigos e etc;
        \item um \textbf{arquivo TODO} nesse diretório, que já foi criado para alguns alunos. Em toda reunião esse arquivo deve ser editado por vocês. Aqueles que mantém uma outra forma deverão atualizar pós reunião, para ser usado na próxima;
        \item ter acesso a pasta \textbf{Comuns} onde colocarei material de interesse de todos.
        \item todos os itens que compartilhar comigo tem que ter seu nome, por exemplo, o nome do projeto do Overleaf, o nome do repositório do GitHub, etc.; não adianta chamar de ``Dissertação de Mestrado'' porque eu não vou saber de quem é.
    \end{itemize}
    \item Você \textbf{pode} ter compartilhado comigo, sendo preferencial que tenha:
    \begin{itemize}
        \item um projeto Overleaf usando um formato de artigo, para escrever um artigo em inglês sobre seu trabalho, com minha ajuda;
        \item um repositório GitHub sincronizado com este projeto;
    \end{itemize}
\end{itemize}


\subsection{Prática de Trabalho}

De forma a possibilitar a nossa colaboração, a prática de trabalho será da seguinte forma:

\begin{itemize}
    \item ao acabar uma sessão de trabalho no Overleaf \textbf{sempre} de commit-\textbf{push} no projeto, isso garantirá que eu não destruo seu trabalho (outros commits podem ser dados durante a sessão);
    \item sempre que acabar uma seção ou subseção de porte, de commit-push com comentário pertinente;
    \item os projetos que estiverem no topo da minha lista terão mais colaboração de minha parte;
    \item eu comentarei usando o pacote \textbf{hacksxexeo}, e corrigirei o que for simples\prof{O ed funciona assim}, vocês também podem comentar;
    \item mantenham o número de \textbf{erros em zero} e o de warnings baixo. Eles aparecem em vermelho (erro) e amarelo (warning). Eu posso ajudar a corrigir quando for estranho para vocês. Se o bibtex gerar erros, vocês provavelmente tem campos faltando na sua base.
    \item eu darei suporte de \LaTeX e bib\TeX, e espero que todos se ajudem;
    \item problemas genéricos de \LaTeX devem ser colocados no grupo de WhatsApp ``\LaTeX ers no PESC''\footnote{\url{https://chat.whatsapp.com/Kad1ZgFQh1RE8e9PAkcBQA}}, para que a solução ajude a todos;
    \item Whatsapp é porta aberta, entrem em contato se quiserem;
    \item perguntas cuja resposta pode favorecer a todos os orientados devem ser feitas no grupo de Orientados do WhatApp;
    \item evitar e-mail para mim;
    \item mantenham o GitHub de código atualizado, se não fizerem vou reclamar.
\end{itemize}

Todos os pacotes \LaTeX tem manual no site CTAN\footnote{\url{https://ctan.org/}}.

\subsection{O que fazer com meus comentários}

Eu farei vários tipos de comentários, que o aluno deve tratar caso a caso. Os comentários coloridos tem duas funções: servir como comentários de coisas a serem feitas e também servir de diálogo. Com o uso a aluna vai entender melhor o que fazer em cada caso.

Seguem algumas instruções gerais, para balizar a decisão:
\begin{itemize}
    \item pedidos de incluir referência, quando atendidos, podem ser apagados;
    \item pedidos de explicação de algo, é melhor colocar um comentário seu logo depois e eu apagar os dois se ficar satisfeito;
    \item comentários que avisam de uma mudança ou que indicam um erro e pedem solução, podem ser apagados se resolvidos. Possivelmente você pode colocar um comentário no lugar dele, se foi algo complicado.
\end{itemize}

\section{Regras que você não pode esquecer - O Plágio!}

Essas são regras importantes que você não pode esquecer:

\begin{itemize}
    \item você \textbf{não pode copiar imagens direto} se não tiver a licença para isso, o que normalmente você não vai ter para imagens de artigos e livros. Você deve redesenhar e indicar a fonte (ver abaixo);
    \item plágio é crime.
\end{itemize}

\section{A escrita}

\subsection{Como escrever a dissertação}

Essa subseção define regras \textbf{obrigatórias} para a escrita da dissertação.

Sua dissertação deve ter 5 ou 7 capítulos, nas seguintes opções, com no máximo 80 páginas de texto.

\begin{itemize}
    \item com sete capítulos
    \begin{enumerate}
        \item Introdução
        \item Revisão da área do problema
        \item Revisão das técnicas de solução
        \item Proposta Teórica
        \item Proposta Técnica, Arquitetura
        \item Experimento 
        \item Conclusão
    \end{enumerate}
    \item com cinco capítulos
    \begin{enumerate}
        \item Introdução
        \item Revisão
        \item Proposta
        \item Experimentos
        \item Conclusão
    \end{enumerate}
\end{itemize}


Os \textbf{tempos de verbo} a serem usados na dissertação são:
\begin{itemize}
    \item a introdução é no presente, como em ``Esse trabalho explora'';
    \item a revisão é no passado quando cita, mas no presente quando analisa e conclui.
    \item a proposta é no presente;
    \item os experimentos são descritos no passado, pois são o resultado de algo já feito, e
    \item a conclusão é no passado, como em ``Esse trabalho explorou'';
\end{itemize}

As citações na forma numérica estão \textbf{terminantemente proibidas}. Não vou nem ler o texto. As citações devem ser na forma ``(XEXÉO, 2021)'', para isso basta eliminar a opção \textbf{numbers} que vem no exemplo.

A dissertação não pode ser escrita na primeira pessoa, nem do singular, nem do plural. O ``nós'' em português, considerado por alguns o plural de modéstia, acabou se tornando o plural majestático. Além disso, a dissertação é apenas suas. Como nenhum de nós é rei ou rainha, não usar. O ``eu'' cabe em algumas situações, mas são muito raras. 

Falamos ``deste trabalho'', com atenção ao uso de ``este'' para indicar o documento propriamente dito, sendo o caso principal do uso do ``este''. Nos outros casos, como a referência a algo que estamos falando, mesmo que na escrita comum seja correto usar ``este'', nós evitamos.

 \textbf{Toda figura e tabela tem que estar citada no texto}, e só pode aparecer a partir da página onde a citação é feita. Isso significa que ela pode aparecer antes da citação se for na mesma página (o que é a opção \verb!t! do ambiente \verb!figure!). O \LaTeX\ faz isso automaticamente, se for usado corretamente.
  
Legendas de tabela são sobre a tabela, de figuras são sob a figura. Você tem que fazer isso na mão, não é automático.

Priorize a forma direta das sentenças: sujeito, verbo, predicado. Se você colocou aspas separando uma oração no meio da sentença, é possível que ela fique mais bem\footnote{Intencionalmente escrevemos essa frase para lembrar que com de verbo no particípio é obrigatório o uso de ``mais bem'' e não o da palavra ``melhor''. Português não é moleza!} colocada em outra posição da sentença.

\subsubsection{A Introdução}

A introdução deve conter \textbf{pelo menos} quatro partes.

\begin{outline}
\1 A \textbf{introdução} propriamente dita, incluindo uma pequena descrição do que foi feito e dos resultados, e uma pequena introdução a metodologia, que será melhor descrita  mais tarde no texto. Deve conter referências e pode indicar partes do texto.
\1 A \textbf{motivação}, em seção específica.
\2 Sempre é interessante que a motivação mostre o \textbf{valor} do resultado da dissertação, seja um valor econômico, seja outro. 
\1 O \textbf{objetivo}, em seção específica, que trato detalhadamente a seguir.
\1 A descrição do documento, com a indicação do conteúdo de cada capítulo\footnote{Eu raramente leio essa parte, mas é tradição tê-la.}.
\end{outline}

Se o candidato teve  publicações relacionadas com o tema da dissertação, deve listá-las também em uma seção adicional. 

Uma regra importante: a dissertação não é um romance, não devemos manter segredos. Pelo contrário, é importante dar os \textit{spoilers} todos, para o leitor entender o que vai ler. Além disso, a conclusão da dissertação deve confirmar que tudo que foi prometido foi alcançado.

\subsection{O objetivo }

Três abordagens são possíveis para o objetivo. 

O primeiro é ele ser uma hipótese a ser provada. 
Nesse caso, novamente se divide em dois tipos, ou a prova é formal, por meio de teoremos, ou deve ser feito um experimento com um hipótese estatística. 
No segundo tipo, é escolhida uma hipótese e feito um teste de hipótese, com a hipótese nula, $H_0$, a que é posta a prova, e a hipótese alternativa, $H_1$.  
No experimento, então, devem ser estabelecidos e aplicados critérios de teste, onde entram conceitos como erro de tipo I, erro de tipo II, p-valor\footnote{Ainda procurando uma referência boa}. Essa é a abordagem mais formal, porém obriga o aluno a ter um experimento definido. É boa, por exemplo, para provar que um algoritmo é melhor que outro. 

O segundo é trabalhar com questões de pesquisa. Esse modelo é 
bastante usado no PESC, principalmente na nossa linha e na linha de Engenharia de Software, e meu preferido atualmente. Para trabalhar com perguntas, é interessante conhecer o método Goal, Question and Metrics\citep{Basili94,gqmm}.
Isso permite uma formalidade no uso das perguntas, que devem ser construídas na forma de um passo a passo, como na estrutura abaixo, criada de forma \textit{ad-hoc} por mim a partir da proposta de tese de doutorado do Luis Felipe Costa:
\begin{enumerate}
    \item Existe um diferença entre a participação de homens e mulheres nas áreas de STEM?
    \item Quais são as dificuldades que criam essas diferenças e as mulheres tem que vencer?
    \item Que ações estão sendo tomadas para evitar essas dificuldades ou auxiliar as mulheres a vence-las?
    \item Que teorias, métodos e ferramentas são usadas nessas ações?
    \item O uso de ambientes EAD com jornadas heroicas pode auxiliar mulheres a vencerem esses desafios?
\end{enumerate}

Nessa estrutura, as primeiras perguntas são feitas para confirmar a existência do problema, as do meio para mostrar qual o estado da arte e a última, mas poderiam ser mais, para mostrar a sua proposta.

O terceiro é ter objetivo e sub-objetivos. É o menos formal, mas também é possível. Eu prefiro que os objetivos sejam transformados em perguntas, mas podemos combinar caso a caso.

Deve ficar claro que os objetivos, ou perguntas, são científicos e não um projeto. Muitos alunos têm um bom projeto, mas não tem uma dissertação de mestrado.

É importante \textbf{nunca usar o termo hipótese se não vai ser feito o teste de hipótese}, porque a banca certamente vai questionar fortemente.

Outros métodos de declara o objetivo da tese são possíveis, e podemos discutir pessoalmente.

\subsubsection{Achando estatísticas gerais}

Muitas vezes na introdução ou na motivação precisamos de dados gerais. A busca pelo Google é uma solução ótima. Partir da Wikipedia também. Algumas instituições que podem ser usadas como referência, como listado a seguir.

\begin{itemize}
    \item ONU e suas ``subsidárias'' como UNESCO, WHO, etc.
    \item Publicações de outros órgãos supra-governamentais, como a União Européia, OCDE, etc.
    \item Informações do governo brasileiro, em sites dos ministérios, autarquias, fundações e no Portal da Transparência\footnote{\url{http://transparencia.gov.br/}}.
    \item World Bank, International Monetary Fund, e similares.
    \item Órgãos oficiais de estatística, como o IBGE.
    \item Empresas de pesquisa reconhecidas, como o IBOPE, Datafolha, etc.
    \item CIA World Factbook.
    \item Gapminder (que aponta outras instituições).
    \item Think Tanks não partidários, como a RAND Corporation.
    \item Think Tanks partidários/sindicais, mas amplamente reconhecido, como DIEESE.
\end{itemize}

\subsection{Dicas para escrever}
    
Você pode escrever do início para o fim se não tem nada, mas se tem resultados é melhor escrever do fim para início, para poupar a escrita da revisão no que é objetivo e foi usado.

Para escrever, cada capítulo e seção tem que ter início, meio e fim. Vocês devem construir um desenho ou delinear em texto o que vão escrever para orientar a escrita, de forma a apresentar e concluir (quase) tudo que fizeram. 
O Método da Pirâmide\citep{minto1,minto2} segue esse princípio de forma \textit{top-down} e é muito eficiente para a criação de textos formais.

Lembrar que o texto tem que ser encadeado. Não dizer ``João disse X. Maria disse Y'', dizer ``João disse X, porém Maria discordou e disse Y''.

Não seja vago. Não se pode afirmar nada sem se apoiar em uma informação factual ou que não seja uma conclusão. Você não pode dizer ``muita gente usa relógio'', tem que dizer ``\verb!\citet{joaozinho2020}! mostra que 60\% da população brasileira usa relógio!''.

Se usar um grupo específico, deixe claro, como em: ``Na preparação deste trabalho observamos as pessoas que entravam no prédio da UFRJ pelo portão principal do bloco A, entre 12:00 e 14:00 do dia 23/4/2021, e 60\% dos 231 observados usava relógio de forma visível''\footnote{O trabalho científico deve ser exato mesmo quando fazemos observações informais.}.

Você não pode achar, tem que concluir racionalmente ou declarar uma premissa explícita, como em ``Esse trabalho parte da premissa que quem tem relógio quer saber as horas''. A premissa pode ser questionada pelo leitor, mas a construção do trabalho é feita sobre a premissa. Já vi bancas onde o membro falou ``discordo da sua premissa, mas sei que seu orientador segue essa linha e vou avaliar considerando ela válida''.

Cuidado ao usar a expressão ``o autor''. Que autor, o do artigo que você discute, ou você? Além disso, você tem certeza que é um autor, não seriam vários, ou uma autora? Por isso é melhor usar a citação, com o comando \verb!\citeauthor{}! ou semelhante\footnote{Devido aos problemas que estamos tendo com o estilo CoppeTeX e o bib\LaTeX, pode ser necessário usar outro comando}. Se forem vários autores, é necessário usar o plural, e o gênero certo. Então é melhor ainda comentar o trabalho e não a pessoa, como em ``esse artigo apresenta a ideia que...''.

Após escrever um parágrafo, leia-o para cortar texto inútil. Menos é mais quando escrevemos em Computação. Não devemos usar adjetivos e advérbios desnecessários, ou que são vagos. Não dizemos ``O grande escritor Jorge Amado'', mas só ``Jorge Amado''. Se necessário, podemos qualificar, como em ``Jorge Amado, escritor bahiano'', principalmente para pessoas menos conhecidas.

 Não faça promessas que não são indicadas onde elas são cumpridas. Ou seja, não escreva ``será apresentado'', mas sim ``o capítulo 5 apresenta''.

Palavras em outras línguas devem ser escritas em itálico.



\subsection{Documentos na web com informações interessantes}

\begin{outline}
\1 O site \href{http://www.contornospesquisa.org/2012/08/como-referenciar-figuras-imagens-e.html}{Contornos - Educação e Pesquisa}\footnote{\url{http://www.contornospesquisa.org/2012/08/como-referenciar-figuras-imagens-e.html}} explica como referenciar figuras;

\1 O documento \href{https://i0.statig.com.br/educacao/trabalhosacademicos/Recomendacoes_elaboracao_monografia.pdf}{Recomendações para elaboração da monografia do trabalho de graduação (TG)}\footnote{\url{https://i0.statig.com.br/educacao/trabalhosacademicos/Recomendacoes_elaboracao_monografia.pdf}} tem algumas recomendações interessantes que servem para a dissertação.
\1 No site \textbf{pós-graduando}
\2 \href{https://posgraduando.com/dez-erros-comuns-na-redacao-cientifica/}{Dez erros comuns na redação científica}\footnote{https://posgraduando.com/dez-erros-comuns-na-redacao-cientifica/}\textsuperscript{,\,}\footnote{Neste documento discordo da regra 5 em geral, mas principalmente no exemplo dado. É de uso corrente dizer que Tabelas apresentam informações.}. 
\2 \href{https://posgraduando.com/dicas-para-melhorar-a-redacao-cientifica/}{Dicas para melhorar a redação científica}\footnote{\url{https://posgraduando.com/dicas-para-melhorar-a-redacao-cientifica/}}
\end{outline}

\subsection{Cuidados ao escrever}

Não confiem em notícias em jornais e revistas, físicos ou digitais, principalmente em português. Jornalistas interpretam errado, traduzem errado, citam fontes não confiáveis cientificamente e etc.
Procurem:
\begin{outline}
    \1 a fonte original da pesquisa;
    \2 se for um relatório pago, procurem resumos ou outras reportagens, ou outra forma de informação que confirme a notícia dada;
    \1 uma reportagem original, normalmente de uma agência de notícias ou da matriz de algumas empresas como CNN e BBC.
\end{outline}

O texto deve ser exclusivamente científico, porém devemos:
\begin{itemize}
    \item ter o cuidado de gerar um texto inclusivo em gênero. Em inglês usar a terceira pessoa do plural evita determinar o gênero. Já em português temos que apelar para construções com ``ele/ela'', voz passiva ou outra;
    \item mesmo tecnicamente, ter cuidado quando falar de religiões, principalmente de forma específica. Um leitor ou um membro da banca mais religioso pode se ofender com comentários ou mesmo com fatos;
    \item ter cuidado com a citação, principalmente a direta, de autores antigos, ou autores reconhecidamente racistas, fascistas, etc,, principalmente sem contextualiza-los no tempo.
    \item ter cuidado com usar fontes não científicas como evidência, por exemplo a Wikipédia. Essas fontes podem ser usadas como curiosidade ou como um passo para atingir fontes interessantes. 
\end{itemize}


\subsection{Tabelas}

Sempre que for apresentar dados sobre mais de uma coisa, use tabelas. Por exemplo, você pode dizer no texto que, em 2021, 8 alunos de mestrado estavam sendo orientados pelo professor, mas não deve dizer algo como ``em 2018, 8 alunos de mestrado, em 2919, em 2020, 7 e em 2020 8 de novo''. Para isso, use uma tabela e um texto ``o levantamento do número de alunos de mestrado do professor ao longo dos quatro últimos anos é demonstrado na Tabela \ref{tab:prof}''.

Também recomendamos usar o estilo \textbf{booktabs}, que fornece comandos \verb|toprule|, \verb|midrule| e \verb|bottomrule| que substituem de forma mais elegante o \verb|hline| em tabelas. A Tabela \ref{tab:prof} usa esses comandos, enquanto a Tabela \ref{tab:tab3} não os usa.

% Please add the following required packages to your document preamble:
% \usepackage{booktabs}
\begin{table}[]
    \centering
    \caption{Uma tabela que usa o estilo \textbf{booktabs} e que deixa os dados corretamente apresentados}
    \label{tab:prof}
    \begin{tabular}{cc}
        \toprule
        Ano  & Alunos \\ \midrule
        2018 & 8      \\
        2019 & 6      \\
        2020 & 7      \\
        2021 & 8      \\ \bottomrule
    \end{tabular}
\end{table}

Apesar muitos textos usarem tabelas com todas as linhas verticais e horizontais, como na Tabela \ref{tab:tab3}, pense em não usá-las, como na Tabela \ref{tab:prof}. \citeauthor{ei} recomenda que devemos dar importância aos dados, não a decoração das tabelas.


\begin{table}[htb]
    \centering
        \caption{Exemplo de uma tabela com todas as bordas, e sem o \textbf{booktabs}.}
\begin{tabular}{|c|c|}
    \hline
    Ano  & Alunos \\ \hline
    2018 & 8      \\ \hline
    2019 & 6      \\ \hline
    2020 & 7      \\ \hline
    2021 & 8      \\ \hline
\end{tabular}

    \label{tab:tab3}
\end{table}

Em alguns lugares se separa \textbf{Tabela} de \textbf{Quadro}, tanto em conteúdo e forma, uma tabela tem dados quantitativos, geralmente números, um quadro tem dados qualitativos, como informações. Na norma, na verdade um documento do IBGE\citep{nat}, as tabelas apresentam bordas laterais abertas, e não precisam de todas as linhas internas. \textbf{Na Coppe}, porém, \textbf{as normas não exigem essa distinção}, e nem citam a existência de quadros. 

\subsection{Livros sobre como escrever}    

\textbf{Não deixe de ler} pelo menos o primeiro capítulo de \citetitle{mw}\citep{mw}.

Uma técnica que eu sugiro é olhar um dos  livros de \citeauthor{minto1}:  \citetitle{minto1}\citep{minto1} ou \citetitle{minto2}\citep{minto2}. 

Uma boa leitura também é \citetitle{wcs}\citep{wcs}.

Sobre como fazer gráficos, diagramas, tabelas, já citei \citetitle{ei}\citep{ei} e recomendo qualquer livro de \citeauthor{ei}.

\subsection{Normas ABNT}

Oficialmente você deve seguir as seguintes normas ABNT, porém usar o estilo CoppeTex\footnote{\url{https://github.com/COPPE-UFRJ/CoppeTeX}} é quase tudo que precisa:
\begin{itemize}
    \item ABNT NBR 6023, Informação e documentação – Referências – Elaboração
\item ABNT NBR 6024, Informação e documentação – Numeração progressiva das seções de um documento
escrito – Apresentação
\item ABNT NBR 6027, Informação e documentação – Sumário – Apresentação
\item ABNT NBR 6028, Informação e documentação – Resumo – Procedimento
\item ABNT NBR 6034, Informação e documentação – Índice – Apresentação
\item ABNT NBR 10520, Informação e documentação – Citações em documentos – Apresentação
\item ABNT NBR 12225, Informação e documentação – Lombada – Apresentação
\item Código de Catalogação Anglo-Americano. 2. ed. rev. 2002. São Paulo: FEBAB, 2004
\item IBGE. Normas de apresentação tabular. 3. ed. Rio de Janeiro, 1993
\end{itemize}

O registro da Coppe verifica apenas as partes não-textuais, e o faz detalhadamente, medindo as distâncias com regras. Internamente você pode ser mais relaxado. 

Todas as normas estão disponíveis no diretório Comuns no Google Drive.
    




\section{O \LaTeX}

\subsection{Pacotes \LaTeX\  obrigatórios}

É obrigatório usar o pacote \textbf{hacksxexeo} com a configuração:

\verb!\usepackage[\usepackage[rascunhos,assuntos,%!
\verb!sugestoes,comentarios]{hacksxexeo}%!


Ele está  disponível em \url{https://github.com/xexeo/hacksxexeo}

Para manter a compatibilidade com o estilo \textbf{CoppeTeX}, nós usaremos o bib\TeX com o pacote natbib, \textbf{porém sem usar a opção numbers na chamada do estilo da Coppe, coisa que todos copiam do exemplo e não percebem.}


\subsection{Estilo hacksxexeo}

O seguinte código deve ser usado no seu texto para comentários coloridos, \prof[possivelmente marcando o texto]{Meu comentário} ou não marcando\cand{seu comentário}, com os comandos:

\verb!\cand[texto a marcar]comentário}! e 

\verb!\candr[texto a marcar]{comentário}{rótulo}!.

Também podem ser colocados textos com cor para ressaltar como sugestão, ou outra coisa (comentar com os comandos anteriores) com os comandos

\verb!\candsug{texto}! e

\verb!\candsug[comentário]{text}!.

Note que a ordem dos parâmetros muda, porque aqui a marcação é obrigatória, e o comentário não. Comandos similares permitem \profrem{remover}   e \proftroca{cambiar}{trocar}   textos.


Deve funcionar \profsug{como estou mostrando nesse parágrafo, aqui com a minha sugestão} e \candsug{aqui como ficaria a sua}.

Esse pacote será aprimorado e vou avisar quando precisar atualizá-lo, por exemplo com \textit{environments} coloridos.

A motivação desse pacote é que nenhum pacote para edição ou colaboração atende minhas vontades arbitrárias.

O arquivo \href{https://github.com/xexeo/hacksxexeo/blob/main/dist/hacksxexeo.pdf}{hacksxexeo.pdf} explica todo o funcionamento do estilo \textbf{hacksxexeo}.


\subsection{Pacotes \LaTeX\  recomendados}
    
    
Os seguintes pacotes são recomendados, para ser usados quando necessários:    
        
\begin{itemize}
    \item \verb!\usepackage{graphicx}!, melhor que o graphics;
    \item \verb!\usepackage{hyperref}!, que \textbf{obrigatoriamente deve ser o último} chamado, para criar links internos, trata urls, além disso ele cria \textit{bookmarks} e facilita muito a vida do leitor. Ele tem um problema: é incompatível com vários outros pacotes. Se algo não funciona como deveria, tente tirar ele e veja se funciona.
    \item \verb!\usepackage{booktabs}! que fornece linhas mais elegantes para as tabelas (\verb!\toprule!,\verb!\midrule!,\verb!\bottomrule!), as Tabelas \ref{tab:tab3} e \ref{tab:prof} mostram a diferença.
    \item se necessário, a família dos pacotes de matemática da AMS, \verb!\usepackage{amsmath}! e
    \verb!\usepackage{amsfonts}!, \verb!\usepackage{amssymb}!, e \verb!\usepackage{amsthm}!.
\end{itemize}

%\begin{table}[htb]
%    \centering
%        \caption{Exemplo de uma tabela normal do \LaTeX .}
%    \begin{tabular}{cc}
%    \hline
%        1 & Exemplo de linhas normais   \\
%        \hline
%        2 & linha 1\\
%        3 & linha 2\\
%        \hline
%    \end{tabular}
%
%    \label{tab:tab2}
%\end{table}
%
%\begin{table}[htb]
%    \centering
%        \caption{Exemplo do uso do pacote \textbf{booktabs} }
%    \begin{tabular}{cc}
%    \toprule
%        1 & Exemplo de linhas mais elegantes   \\
%        \midrule
%        2 & linha 1\\
%        3 & linha 2\\
%        \bottomrule
%    \end{tabular}
%
%    \label{tab:tab1}
%\end{table}

\subsubsection{Pacotes não recomendados e alternativas}

\begin{outline}
    \1 \textbf{subfigure} foi deprecado, usar o \textbf{subfig}.
    \1 \textbf{ulem} dá problema com as referências da Coppe, usar o \textbf{soulutf8}, que já está habilitado no \textbf{hacksxexeo}
\end{outline}

\subsection{Dicas de \LaTeX}

\begin{outline}
\1 O LaTeX já entende caracteres acentuados, não use os comandos como \verb|\´|
\1 Dicas de organização do documento \LaTeX:
\2 Escreva cada frase do parágrafo em uma linha do arquivo, isso facilitará encontrar os erros.
\2 Não quebre muito o seu arquivo se não for necessário. Capítulos pequenos não precisam ter várias partes.
\2 Numere os nomes dos arquivos de capítulos e seções de forma que fiquem na sequência certa no diretório do Overleaf, como em: 00Introducao.tex, 01Motivacao.tex, 02OProblema.tex.
\1 Dicas de \LaTeX
\2 Temos um grupo de Whatsapp para dúvidas de \LaTeX\footnote{\url{https://chat.whatsapp.com/Kad1ZgFQh1RE8e9PAkcBQA}}, por favor façam essas perguntas por lá, para favorecer a todos. Entrem no grupo!
\2 Não compliquem o uso do \LaTeX, não se percam com detalhes, peçam ajuda. 
\2 As figuras e tabelas, e todos os flutuantes, do \LaTeX\  flutuam mesmo, \textbf{não se preocupe} até o momento final da impressão, onde podemos tentar resolver. 
\3 Se quiser proibir que figuras e tabelas mudem de seção use o pacote \verb!\usepackage[section]{placeins}!, também aconselho a usar sempre \verb!\begin{figure}[hbt]!, que tenta colocar no local, no fim da página ou no topo. 
\3 Não gosto da opção \verb!H! do pacote \textbf{float}, porque pouco resolve o problema e às vezes gera outros. Só recomendo usar na edição final para entregar, se quiser resolver algum problema mais grave. Sempre evite usar o \verb!H!.
\2 Aspas não são feitas com \verb!"! mas sim com \verb!``! no início e \verb!''! no final.
\3 Não consigo gerar estes caracteres! São o acento grave e o apóstrofe reto, repetidos duas vezes. No teclado ABNT estão do lado esquerdo do 1 e do lado direito do P (com shift). 
\2 Controle o tamanho das imagens com as opções do comando \textbf{includegraphics}, como em:

 \verb!\includegraphics[width=0.5\linewidth]{arquivo}!, 
 
 \verb!\includegraphics[scale=0.5]{arquivo}!, 
 
 ou outros comandos do pacote \textbf{graphicx}.

\2 Cuidado com as imagens de baixa definição, e também com as de tamanho muito grande, que ao serem reduzidas podem perder detalhes. 
\2 Desenhos vetoriais funcionam bem no \LaTeX, e existe um pacote bem complexo e difícil de usar chamado \textbf{tikz} que permite desenhar com comandos. 
\3 Imagens não vetoriais podem perder detalhes, verifique o pdf e a impressão.
%\2 Usando o pacote \textbf{hacksxexeo} é possível cortar um texto! \candrem{Por exemplo, cortei esse aqui.} O uso dos comandos desse pacote ajuda a saber quem fez o que.)
\2 Use um gerador de tabelas online para começar a criá-las, como o \href{https://www.tablesgenerator.com/}{Tables Generator}\footnote{\url{https://www.tablesgenerator.com/}}
\end{outline}

O estilo CoppeTeX já vem, em seu exemplo, com uma lista de símbolos. Para definir um símbolo é necessários seguir o seguinte exemplo:

\begin{verbatim}
\symbl{$\mathbb{R}$}{Conjunto dos números reais}
\end{verbatim}


\subsection{Errors e Warnings}

Um erro não pode ser deixado no programa, porque não sabemos o que ele pode produzir. Mesmo que a saída não esteja sendo alterada, mais tarde isso pode não ser verdade.

Warnings podem ser deixados, se compreendemos o que está acontecendo.

Mensagens de \textit{Overfull}, que são azuis no Overleaf, indicam que provavelmente alguma coisa saiu das bordas laterais. Como o valor é em pontos, e um ponto equivale a um pouco mais de  $0.035$ centímetros, erros pequenos provavelmente não são vistos.

Mensagens \textit{Underfull}, também azuis no Overleaf, normalmente significam que tem um buraco na sua diagramação. Novamente, pela medida podemos pensar se devemos nos preocupar ou não.

Uma inspeção visual nas bordas e na diagramação geral do documento rapidamente indicam onde estão os problemas mais importantes.

Nesse documento, por exemplo, a url que aponta o documento sobre como referenciar figuras deu um erro de $130.28$ pontos, aproximadamente $4.59$ cm, o que provocou que ela saísse não só das bordas da linha, mas também da borda do papel.


\section{Ajuda!}
O que você gostaria de ver nesse documento? 


\section{Agradecimento}
Agradeço aos alunos que colaboraram revisando este documento. Ele está disponível em: 

\printbibliography

\end{document}
