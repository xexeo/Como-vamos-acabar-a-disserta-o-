\documentclass{article}
\usepackage[T1]{fontenc}
\usepackage[brazilian]{babel}
\usepackage[show]{ed}
\usepackage{hyperref}

\title{Como vamos acabar a dissertação?}
\author{Geraldo Xexéo}
\date{April 2021 - Versão 0.2}

\begin{document}

\maketitle

\section{Introdução}

Esse documento apresenta as regras para os alunos de 2017/1 que vão tentar acabar sua dissertação de mestrado, aproveitando o perdão da pandemia.

\section{Estrutura de troca de informações}

\begin{itemize}
    \item Você precisa obrigatoriamente ter compartilhado comigo:
    \begin{itemize}
        \item um projeto Overleaf usando o formato CoopeTex. Esse arquivo pode ser criado por mim para facilitar o compartilhamento;
        \item um repositório GitHub sincronizado com esse projeto, criado a partir do Overleaf, novamente criado por mim;
        \item um repositório de código, criado por você, no GitHub (eu prefiro) ou no GitLab;
        \item um diretório no Google Drive, criado por mim, para troca de artigos e etc;
        \item um arquivo TODO nesse diretório, que já foi criado para alguns alunos. Em toda reunião esse arquivo deve ser editado por vocês. Aqueles que mantém uma outra forma deverão atualizar pós reunião, para ser usado na próxima.
    \end{itemize}
    \item Você pode ter compartilhado comigo, sendo preferencial que tenha:
    \begin{itemize}
        \item um projeto Overleaf usando um formato de artigo, para escrever um artigo em inglês sobre seu trabalho, com minha ajuda;
        \item um repositório GitHub sincronizado com este projeto;
    \end{itemize}
    \item São obrigatórios as seguintes configurações nos projetos Overleaf:
    \begin{itemize}
        \item \textbf{fontencode} com a configuração \verb!\usepackage[T1]{fontenc}!
        \item \textbf{babel} com a configuração \verb!\usepackage[english, brazilian]{babel}! para textos em português ou \verb!\usepackage[brazilian,english]{babel}! para textos em inglês. Sim, é nessa ordem.
        \item \textbf{ed} com a configuração \verb!\usepackage[show]{ed}!
        \item usar o processador \textbf{LuaLaTeX}, que é mais poderoso, configuração no Menu do projeto, e dispensa configurar para usar utf-8, que é natural dele.
        \item usar o \textbf{bibLaTeX} e não o bibTeX, configuração no Menu do projeto.
    \end{itemize}
\end{itemize}




\section{O que será avaliado}

O andamento será da seguinte forma:
\begin{itemize}
    \item toda semana vamos nos encontrar, a menos de problemas pessoais;
    \item o texto da dissertação tem que evoluir toda a semana, a avaliação básica será sobre essa evolução;
    \item toda semana terá um acordo básico do que deve ser feito, que, mesmo que não explicitado, inclui evoluir o texto da dissertação;
    \item espera-se que o acordo seja cumprido, porém entende-se que alguns podem ser mais longos que 1 semana, e deverá ser cumprido parcialmente.
\end{itemize}

\section{Prática de Trabalho}

A prática de trabalho será da seguinte forma:

\begin{itemize}
    \item ao acabar uma sessão de trabalho no Overleaf \textbf{sempre} de commit-\textbf{push} no projeto, isso garantirá que eu não destruo seu trabalho (outros commits podem ser dados durante a sessão);
    \item sempre que acabar uma seção ou subseção de porte, de commit-push com comentário pertinente;
    \item os projetos que estiverem no topo da minha lista terão mais colaboração de minha parte;
    \item eu comentarei usando o pacote \textbf{ed}, e corrigirei o que for simples\ednote{O ed funciona assim}\edissue{Com notas e issues};
    \item mantenham o número de \textbf{erros em zero} e o de warnings baixo. Eles aparecem em vermelho (erro) e amarelo (warning). Eu posso ajudar a corrigir quando for estranho para vocês.
    \item eu darei suporte de \LaTeX e bib\LaTeX, e espero que todos se ajudem;
    \item Whatsapp é porta aberta, entrem em contato se quiserem;
    \item perguntas cuja resposta pode favorecer a todos devem ser feitas no grupo de Orientados do WhatApp;
    \item evitar e-mail;
    \item mantenham o GitHub de código atualizado, se não fizerem vou reclamar.
\end{itemize}

Todos os pacotes tem manual no cite CTAN\footnote{\url{https://ctan.org/}}.

\section{Regras que você não pode esquecer}

Essas são regras importantes que você não pode esquecer:

\begin{itemize}
    \item você \textbf{não pode copiar imagens direto} se não tiver a licença para isso, o que normalmente você não vai ter para imagens de artigos e livros. Você deve redesenhar e indicar a fonte (ver abaixo);
    \item plágio é crime.
\end{itemize}



\section{Pacotes \LaTeX recomendados}

\begin{itemize}
    \item \verb!\usepackage{graphicx}!, melhor que o graphics;
    \item \verb!\usepackage{hyperref}!, que \textbf{obrigatoriamente deve ser o último} chamado;
\end{itemize}

\section{Dicas}

O site \url{http://www.contornospesquisa.org/2012/08/como-referenciar-figuras-imagens-e.html} explica como referenciar figuras

Temos um grupo de Whatsapp para dúvidas de \LaTeX\footnote{\url{https://chat.whatsapp.com/Kad1ZgFQh1RE8e9PAkcBQA}}, por favor façam essas perguntas por lá, para favorecer a todos. Entrem no grupo!

Não compliquem o uso do \LaTeX, não se percam com detalhes, peçam ajuda. 

As figuras do \LaTeX flutuam mesmo, não se preocupe. Se quiser proibir que elas mudem de seção use o pacote \verb!\usepackage[section]{placeins}!.

Aspas não são feitas com \verb!"! mas sim com \verb!``! no início e \verb!''! no final.

Controle o tamanho das imagens com \verb!\includegraphics[width=0.5\linewidth]{arquivo}! ou \verb!\includegraphics[scale=0.5]{arquivo}! ou outros comandos do pacote \textbf{graphicx}.

Cuidado com as imagens de baixa definição.

\section{Normas ABNT}

Oficialmente você deve seguir as seguintes normas ABNT, porém usar o estilo CoppeTeX é quase tudo que precisa:
\begin{itemize}
    \item ABNT NBR 6023, Informação e documentação – Referências – Elaboração
\item ABNT NBR 6024, Informação e documentação – Numeração progressiva das seções de um documento
escrito – Apresentação
\item ABNT NBR 6027, Informação e documentação – Sumário – Apresentação
\item ABNT NBR 6028, Informação e documentação – Resumo – Procedimento
\item ABNT NBR 6034, Informação e documentação – Índice – Apresentação
\item ABNT NBR 10520, Informação e documentação – Citações em documentos – Apresentação
\item ABNT NBR 12225, Informação e documentação – Lombada – Apresentação
\item Código de Catalogação Anglo-Americano. 2. ed. rev. 2002. São Paulo: FEBAB, 2004
\item IBGE. Normas de apresentação tabular. 3. ed. Rio de Janeiro, 1993
\end{itemize}

\section{Ajuda!}

O que você gostaria de ver nesse documento? 

\end{document}
